%%=============================================================================
%% Voorwoord
%%=============================================================================

\chapter*{\IfLanguageName{dutch}{Woord vooraf}{Preface}}%
\label{ch:voorwoord}

%% TODO:
%% Het voorwoord is het enige deel van de bachelorproef waar je vanuit je
%% eigen standpunt (``ik-vorm'') mag schrijven. Je kan hier bv. motiveren
%% waarom jij het onderwerp wil bespreken.
%% Vergeet ook niet te bedanken wie je geholpen/gesteund/... heeft



Ik wil dit onderwerp bespreken omdat Garbage Collection in Java wel een breed onderwerp is dat nog steeds relevant blijkt tot vandaag de dag, het leek mij ook interessant omdat dit thema toch ook wel zeer diepgaand is, in de zin dat het zeer \textit{low-level} gaat, en ook een groot deel theoretische computerwetenschap is.
Verder vind ik het zeer interessant om over de mainframe te kunnen spreken, het is een grootschalig thema dat vandaag de dag eerder in de schaduw ligt.
Zelf ben ik zeer blij dat ik in de mainframe niche ben getrokken, het is een rijke en boeiende wereld waarin ik nog veel te leren heb.



Vanuit HOGENT zou ik graag mijn promoter, Sion Verschraege bedanken voor de telkens spoedige feedback en beschikbaarheid, en Leendert Blondeel voor het oprichten van het mainframe expert keuzetraject.


Ik wil mijn co-promoter, Jan Cannaerts bedanken voor de hulp in het mogelijk maken om de Java programma's te kunnen laten werken op mainframe, en voor de theoretische uitleg hieromtrent.
Ook wil ik Solidaris bedanken voor de mogelijkheid om hun mainframes te kunnen gebruiken om mijn testen uit te voeren.


Ik zou ook graag mijn vader, Johan Briké, mijn moeder Jadranka Briké, en mijn vriendin Vaneeda Vermeersch bedanken voor hun steun.

Als laatste zou ik graag Mathias De Gryse en Michael Liu bedanken om mij hun hulp te bieden op taalkundig vlak.



%%=============================================================================
%% Samenvatting
%%=============================================================================

% TODO: De "abstract" of samenvatting is een kernachtige (~ 1 blz. voor een
% thesis) synthese van het document.
%
% Een goede abstract biedt een kernachtig antwoord op volgende vragen:
%
% 1. Waarover gaat de bachelorproef?
% 2. Waarom heb je er over geschreven?
% 3. Hoe heb je het onderzoek uitgevoerd?
% 4. Wat waren de resultaten? Wat blijkt uit je onderzoek?
% 5. Wat betekenen je resultaten? Wat is de relevantie voor het werkveld?
%
% Daarom bestaat een abstract uit volgende componenten:
%
% - inleiding + kaderen thema
% - probleemstelling
% - (centrale) onderzoeksvraag
% - onderzoeksdoelstelling
% - methodologie
% - resultaten (beperk tot de belangrijkste, relevant voor de onderzoeksvraag)
% - conclusies, aanbevelingen, beperkingen
%
% LET OP! Een samenvatting is GEEN voorwoord!

%%---------- Nederlandse samenvatting -----------------------------------------
%
% TODO: Als je je bachelorproef in het Engels schrijft, moet je eerst een
% Nederlandse samenvatting invoegen. Haal daarvoor onderstaande code uit
% commentaar.
% Wie zijn bachelorproef in het Nederlands schrijft, kan dit negeren, de inhoud
% wordt niet in het document ingevoegd.

\IfLanguageName{english}{%
\selectlanguage{dutch}
\chapter*{Samenvatting}
%todo
\selectlanguage{english}
}{}

%%---------- Samenvatting -----------------------------------------------------
% De samenvatting in de hoofdtaal van het document

\chapter*{\IfLanguageName{dutch}{Samenvatting}{Abstract}}
%inleiding
Deze bachelorproef gaat over de impact op Java Garbage Collection wanneer een Java applicatie gemigreerd wordt van niet-mainframe naar mainframe.
Hedendaags is er nog geen publiek openbaar onderzoek hiernaartoe gedaan.
Terwijl dit wel een interessante thema is, zeker voor bedrijven die geïnteresseerd zijn in het migreren van hun Java applicaties naar mainframe.



%probleemstelling
Een mainframe is een zeer krachtige computer, gebruikt bij bedrijven die grootschalig hoeveelheden data moeten verwerken.
Programma's op mainframe maken gebruik van verouderde programmeertalen, er is een tekort aan werknemers die hiermee bekwaam zijn.
Daarom dat het interessant is om Java op mainframe te gebruiken, zowel om de kracht van de mainframe te kunnen gebruiken, en omdat een groter hoeveelheid van de arbeidsmarkt hiermee bekwaam is.

%onderzoeksvraag

%methodologie
Voor de Java applicaties zelve maken we gebruik van de Renaissance benchmark suite, deze suite geeft ons een uitgebreid aantal grootschalige en verschillende Java applicaties.
We voeren vier van deze programma's uit op zowel niet-mainframe als mainframe, met een vijftal Garbage Collection policies.

%resultaten


%conclusie
Uit deze resultaten kunnen we concluderen dat migratie vanuit niet-mainframe naar mainframe wel degelijk een impact heeft op Garbage Collection in Java.
Hiermee kunnen we aanbevelen dat indien er gemigreerd wordt van niet-mainframe naar mainframe, of vice-versa, er opnieuw getest moet worden naar wat de optimaalste Garbage Collection policy is.





%todo

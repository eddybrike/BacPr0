%%=============================================================================
%% Inleiding
%%=============================================================================

\chapter{\IfLanguageName{dutch}{Inleiding}{Introduction}}%
\label{ch:inleiding}

%De inleiding moet de lezer net genoeg informatie verschaffen om het onderwerp te begrijpen en in te zien waarom de onderzoeksvraag de moeite waard is om te onderzoeken. In de inleiding ga je literatuurverwijzingen beperken, zit de tekst vlot leesbaar blijft. Je kan de inleiding verder onderverdelen in secties als dit de tekst verduidelijkt. Zaken die aan bod kunnen komen in de inleiding~\autocite{Pollefliet2011}:


\section{\IfLanguageName{dutch}{Probleemstelling}{Problem Statement}}%
\label{sec:probleemstelling}

%Uit je probleemstelling moet duidelijk zijn dat je onderzoek een meerwaarde heeft voor een concrete doelgroep. De doelgroep moet goed gedefinieerd en afgelijnd zijn. Doelgroepen als ``bedrijven,'' ``KMO's'', systeembeheerders, enz.~zijn nog te vaag. Als je een lijstje kan maken van de personen/organisaties die een meerwaarde zullen vinden in deze bachelorproef (dit is eigenlijk je steekproefkader), dan is dat een indicatie dat de doelgroep goed gedefinieerd is. Dit kan een enkel bedrijf zijn of zelfs één persoon (je co-promotor/opdrachtgever).


\let\oldquote\quote
\let\endoldquote\endquote
\renewenvironment{quote}[2][]
{\if\relax\detokenize{#1}\relax
    \def\quoteauthor{#2}%
    \else
    \def\quoteauthor{#2~---~#1}%
    \fi
    \oldquote}
{\par\nobreak\smallskip\hfill(\quoteauthor)%
    \endoldquote\addvspace{\bigskipamount}}

Bedrijven die gebruik maken van een mainframe kunnen baat hebben bij het maken van Java programma's voor mainframe, of het migreren van bestaande Java programma's naar mainframe.
Er zijn meerdere redenen waarom dit een voordeel kan bieden, de voornaamste zijn:
    \begin{itemize}
    \item Een mainframe is gemaakt om zeer veel transacties te kunnen verwerken, Java programma's die grote hoeveelheden data moeten verwerken kunnen dus de mainframe hiervoor gebruiken.
    \item De mainframe maakt gebruik van technologieën en programmeertalen die al ontworpen zijn sinds de eerste generaties van mainframes.
     Zoals \textit{COBOL} uit 1959 of \textit{Job Control Language} uit 1964 die beschreven werd door de uitvinder als;
    \begin{quote}{Fred Brooks}
     ``\textit{Job Control Language is the worst programming language ever designed anywhere by anybody for any purpose.}''\autocite{Brooks2004}
    \end{quote}}
    
    

    Een voordeel van de mainframe is dat het compatibel blijft met oude programma's, ongeacht hoe oud of welke model mainframe er gebruikt wordt.
    Dit zorgt ook voor een nadeel ervan.
    Omdat het compatibel blijft met zelfs de oudste programma's die geschreven werden in onder andere \textit{PL/I}, \textit{COBOL}, \textit{Assembler} of \textit{CICS}, zien we dat nieuwe programma's voor de mainframe ook in deze talen geschreven worden.
    
    
    Het nadeel hiervan is dat de hedendaagse hoeveelheid mensen in de arbeidsmarkt bekwaam in deze talen laag is en blijft dalen.
    
    Daarom dat Java op mainframe gebruiken interessant is omdat er een groter aantal mensen bekwaam zijn met deze programmeertaal in vergelijking met de traditionele mainframe programmeertalen.
    
    
  
    
    \item Java applicaties op mainframe worden door de zIIP processor uitgevoerd.
    Deze processor is specifiek gemaakt om, onder andere, Java applicaties uit te voeren.
    Hierdoor zouden de Java applicaties nog beter presteren, in vergelijking met het uitvoeren door de mainframe CPU's.
    
    \item De kost van de gebruikte processor verwerkingskracht van programma's via de mainframe CPU's gemeten in \textit{Million service units} (MSU).
    Er worden extra kosten aangerekend door IBM aan de hand van de hoeveelheid MSU er verbuikt worden.
    
    
    Programma's die via de zIIP processor uitgevoerd worden, dus onder andere Java applicaties, worden niet mee berekend voor de hoeveelheid gebruikte MSU.
    Hierdoor betekend het dat het gebruiken van de zIIP processor, en dus Java applicaties, ook financieel interessant is voor een bedrijf.
    
    
    
 \end{itemize}
\section{\IfLanguageName{dutch}{Onderzoeksvraag}{Research question}}%
\label{sec:onderzoeksvraag}

%Wees zo concreet mogelijk bij het formuleren van je onderzoeksvraag. Een onderzoeksvraag is trouwens iets waar nog niemand op dit moment een antwoord heeft (voor zover je kan nagaan). Het opzoeken van bestaande informatie (bv. ``welke tools bestaan er voor deze toepassing?'') is dus geen onderzoeksvraag. Je kan de onderzoeksvraag verder specifiëren in deelvragen. Bv.~als je onderzoek gaat over performantiemetingen, dan 
De onderzoeksvraag van deze thesis is; 'Wat zijn de gevolgen voor Garbage Collection bij het migreren van Java programma's naar mainframe?'.
Hieruit kunnen we ook enkele deelvragen stellen.

\begin{itemize}
    \item Is Garbage Collection voor Java op niet-mainframe en mainframe gelijkaardig qua performantie?
    \item Wat is de invloed van Garbage Collection in Java op niet-mainframe en op mainframe qua efficiëntie?
    \item Werkt een Garbage Collector qua performantie hetzelfde wanneer er een programma van niet-mainframe naar mainframe migreert wordt?
\end{itemize}



  

\section{\IfLanguageName{dutch}{Onderzoeksdoelstelling}{Research objective}}%
\label{sec:onderzoeksdoelstelling}

De doelstelling van dit onderzoek is om een vergelijkende studie te maken omtrent de werking van Garbage Collection op mainframe en niet-mainframe.
Hierdoor zouden we een aanbeveling kunnen geven i.v.m. hoe men moet omgaan met Garbage Collection bij het migreren naar mainframe.
Dit door het vergelijken van een aantal statistieken verkregen vanuit testprogramma's.

%Wat is het beoogde resultaat van je bachelorproef? Wat zijn de criteria voor succes? Beschrijf die zo concreet mogelijk. Gaat het bv.\ om een proof-of-concept, een prototype, een verslag met aanbevelingen, een vergelijkende studie, enz.

\pagebreak
\section{\IfLanguageName{dutch}{Opzet van deze bachelorproef}{Structure of this bachelor thesis}}%
\label{sec:opzet-bachelorproef}

% Het is gebruikelijk aan het einde van de inleiding een overzicht te
% geven van de opbouw van de rest van de tekst. Deze sectie bevat al een aanzet
% die je kan aanvullen/aanpassen in functie van je eigen tekst.

De rest van deze bachelorproef is als volgt opgebouwd:

In Hoofdstuk~\ref{ch:stand-van-zaken} wordt een overzicht gegeven van de stand van zaken binnen het onderzoeksdomein, op basis van een literatuurstudie.

In Hoofdstuk~\ref{ch:methodologie} wordt de methodologie toegelicht en worden de gebruikte onderzoekstechnieken besproken om een antwoord te kunnen formuleren op de onderzoeksvragen.

% TODO: Vul hier aan voor je eigen hoofstukken, één of twee zinnen per hoofdstuk

In Hoofdstuk~\ref{ch:uitvoering} wordt de effectieve uitvoering van de testen besproken.

In Hoofdstuk~\ref{ch:analyse} wordt de bekomen data gegeven en toegelicht.

In Hoofdstuk~\ref{ch:conclusie}, tenslotte, wordt de conclusie gegeven en een antwoord geformuleerd op de onderzoeksvragen. Daarbij wordt ook een aanzet gegeven voor toekomstig onderzoek binnen dit domein.